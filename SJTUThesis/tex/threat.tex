%# -*- coding: utf-8-unix -*-
%%==================================================
%% chapter02.tex for SJTU Master Thesis
%% based on CASthesis
%% modified by wei.jianwen@gmail.com
%% Encoding: UTF-8
%%==================================================

\chapter{威胁模型}
\label{s:threat_model}

目前在互联网上,不受信任的应用的数量正在快速增长,让不受信任的应用程序能够更好地运行在系统中,是一件相当困难的事情。这些不受信用的应用程序可能隐藏有带有攻击意图的代码,它们获取操作系统高权限、读取文件系统中的敏感数据(比如密码、照片、文档等)、植入广告、导致系统崩溃等等。而攻击手段有很多,其中包括代码注入、Cross Site Scripting(CSS)、缓冲区溢出、Return-oriented programming(ROP) 等等 \parencite{miwa}。	

传统的操作系统并没有很好地解决这方面的问题,在隔离方面,仅仅依靠运行时的简单隔离是不够的。新的趋势使得操作系统应该在处理器,内存等硬件层面和进程,文件系统等软件层面进行更加细致的隔离和容错。这也是沙箱技术关注的焦点。不同的沙箱技术针对的威胁模型都是不一样的,但是它们在宏观上都有一个共同点:防止不受信任的应用程序可以随意访问地操作系统或者底层的硬件。
